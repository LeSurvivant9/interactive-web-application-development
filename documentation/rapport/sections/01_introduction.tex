\section{Introduction et contexte}\label{sec:introduction}

Ce document présente le rapport de réalisation du projet \textbf{MediaTracker}, développé dans le cadre du module SAE \enquote{Développement d'application web interactive} (FISA S5).
Au-delà de l'exercice académique, ce projet incarne une volonté de concevoir une solution professionnelle et pérenne répondant à une problématique quotidienne de consommation culturelle.

\subsection{Cadre du projet et problématique}\label{subsec:cadre_projet}

À l'ère du numérique, notre consommation de divertissement est fragmentée sur de multiples plateformes (Netflix, Steam, Spotify, Kindle).
Il devient complexe de conserver un historique centralisé de notre vie culturelle ou de gérer simplement une liste d'envies sans basculer d'une application à l'autre.
L'ambition de \textit{MediaTracker} est de devenir ce point de convergence unique : un véritable \enquote{compagnon médiatique}.

Si la vision à long terme inclut la gestion des livres, des jeux vidéo et de la musique, cette première version (v1) se concentre spécifiquement sur l'univers audiovisuel (films et séries).

\subsection{Objectifs pédagogiques et personnels}\label{subsec:objectifs}

Ce projet représente un jalon important dans mon parcours d'ingénieur pour plusieurs raisons :
\begin{itemize}
  \item \textbf{Montée en compétences techniques :} L'objectif est de maîtriser des technologies de pointe, notamment l'implémentation complète d'une API GraphQL (découverte totale) et l'utilisation des dernières avancées de Next.js 16.
  \item \textbf{Professionnalisation :} Passer du statut de \enquote{projet étudiant} à celui de \enquote{produit portfolio} complet, respectant les standards de l'industrie (architecture propre, typage strict, CI/CD).
  \item \textbf{Validation des acquis :} Démontrer ma capacité à mener un projet Full-Stack complexe en autonomie, de la conception de la base de données jusqu'au déploiement.
\end{itemize}

\subsection{Périmètre fonctionnel}\label{subsec:perimetre}

Pour cette première itération, les fonctionnalités ont été priorisées pour garantir une expérience utilisateur fluide et aboutie :
\begin{itemize}
  \item \textbf{Gestion de collection :} Recherche via l'API TVDB, ajout à la bibliothèque et suppression d'œuvres.
  \item \textbf{Suivi de progression :} Marquage des épisodes vus et suivi global des séries.
  \item \textbf{Fonctions sociales :} Système de notation et de commentaires pour chaque média.
\end{itemize}

\begin{figure}[H]
  \centering
  \includesvg[width=0.3\textwidth]{context}
  \caption{Diagramme de contexte (niveau 0)}\label{fig:context}
\end{figure}

\subsection{Méthodologie et organisation}\label{subsec:methodologie}

La gestion de ce projet en solo a nécessité une rigueur organisationnelle stricte pour tenir les délais impartis.
J'ai adopté une approche itérative inspirée des méthodes SCRUM :
\begin{itemize}
  \item \textbf{Planification par sprints :} Le travail a été découpé en blocs fonctionnels distincts (infrastructure, backend core, UI integration), visibles sur le diagramme de Gantt (Figure~\ref{fig:gantt}).
  \item \textbf{Processus de qualité :} Chaque fin de cycle inclut une phase de\enquote{Self-Review} et de refactoring, garantissant que la dette technique ne s'accumule pas.
  \item \textbf{Convention de code :} L'application stricte des \textit{Conventional Commits} permet une traçabilité complète des évolutions et des correctifs.
\end{itemize}

\begin{figure}[H]
  \centering
  \includesvg[width=\textwidth, inkscapelatex=false]{gantt}
  \caption{Planning prévisionnel et découpage en sprints}\label{fig:gantt}
\end{figure}